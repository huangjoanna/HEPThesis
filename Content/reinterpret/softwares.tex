\section{\contur and related software tools}
\label{sec:reinterpretsoftware}In order to assess the plausibility of a BSM model in light of the existing measurements a couple of ingredients are necessary. First, the BSM model must be interfaced to a Monte Carlo event generator in order to simulate events. Facilitating this procedure is the \FeynRules~\cite{}. At its core, \FeynRules is a mathematica package used to develop BSM models. Upon the input of a Lagragian, it derives the Feynman rules. An important feature is the export to the Universal FeynRules Object~\cite{ufo} (UFO) format, compatible with a range of event generators. UFO files hold the basic information about new particles and parameters which are directly used to generate events for new physics signals. The MC event generator used for studies in this thesis is \herwig~\cite{herwig7}, the default MC event generator used by \contur. 

Second, a platform to compare the simulated BSM events to the plethora of LHC measurements is needed. This job is done by \rivet~\cite{rivet}, a general-purpose tool used to reproduce analysis procedures on simulated events. It hosts a wealth of measurements from various high energy physics experiments. Each measurement is written into a \rivet routine which preserves the workflow of the analysis. Included are crucial information such as the fiducial region selection cuts and observable binnings. With this information, \rivet is able to transform the output from Monte Carlo event generators into histograms of cross-sections in the scope of the measurement. In general, it is considered good practice for \LHC measurements to publish their data on \HEPData~\cite{HEPData} and to provide the analysis code as a \rivet routine. \rivet's growing library of available analyses and it's ability to make fast and easy to comparisons between raw generator output and particle-level data is the foundation upon which \contur is built. 

The \contur workflow is summarized in Figure~\ref{fig:conturworkflow} in four steps, taken from Reference~\cite{conturmanual}. Each step is written in bold, with the tool used to achieve it in smaller font. First in the workflow is defining a parameter space (also called a grid) to perform the sensitivity scan for the chosen BSM model. Once defined, \contur's built in function will generate individual run scripts for each point in the parameter space and submit them to an HPC cluster. Each run script calls upon \herwig to generate events for the BSM model at that run point. \herwig is interfaced to \rivet directly so the events are directly piped into the observables' histograms. The next step is to evaluate the likelihood for the BSM model (at each run point) using the physics observables as inputs to a statistical analysis. This is \contur's main functionality. Finally, \contur's plotting functions allow the user to visualize the likelihood evaluation output of their scanned parameter space. An in depth description of each step can be found in Reference~\cite{conturmanual}.