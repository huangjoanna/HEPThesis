\chapter{\mFourL: A measurement designed for re-interpretation}
\label{chap:fourlepton}

\chapterquote{Very inspiring quote}
{Very inspiring quote author}

%% m4l motivation
\section{Motivation for the \mFourL measurement}
\label{sec:fourlepmotivation}

The four lepton channel is a particularly interesting channel to study as it receives contributions from many physics processes. First and foremost, there is the production of a pair of \PZ bosons via quark-antiquark interactions in \info[]{s-channel not in SM because it includes neutral ZZZ or ZZ\photon vertex} both the \Ptop- and \Pup-channel. The \Ptop-channel diagram is shown in Figure \ref{fig:m4lfeynman}, and represents, by far, \improvement[]{Read more baout why the u-channel diagram is not preferred} the largest contribution to the \mFourL distribution. Second in magnitude is the gluon-induced production of a \PZ boson pair. 

The idea behind this analysis is to make a measurement as inclusive and as model-independent as possible. The fiducial region definition follows closely the acceptance of the detector. Furthermore, by loosening the mass cuts, there is higher event acceptance especially in the low mass regions. Preliminary studies were conducted to investigate the impact of loosening and simplifying the dilepton lower mass cut to \unit{5}{\GeV} and removing the upper mass cut, for example, as opposed to the varying higher cuts in the previous round of the analysis. Unsurprisingly, these result in a higher event yield in both the low and high mass tails of the \mFourL distribution. 
\missingfigure{Emily plots for loosening mass cuts}

% this channel provides a clean leptonic final state resulting in a small instrumental background, where one or more of the reconstructed lepton candidates originate from the misidentification of jet fragments or from nonprompt leptons.

%% Theoretical predictions 
\section{Theoretical predictions}
\label{sec:theory}

Theoretical predictions from Monte Carlo simulations

%% Signal definition and event selection
\section{Signal definition and event selection}
\label{sec:signaldef}
\subsection{Lepton definitions}

For particle physicists, a prompt lepton simply means the lepton did not originate from a hadron. Prompt leptons are further classified into three categories depending on their association with emitted photons. These three categories are:
\begin{itemize}
    \item Born leptons: leptons prior to QED Final State Radiation (FSR);
    \item Bare leptons: leptons after QED FSR;
    \item Dressed leptons: leptons after QED FSR, that then have the four momenta of nearby radiated photons added to it. 
\end{itemize}
The ATLAS detector makes lepton measurements after QED FSR has occurred. It is for this reason that born leptons are not the best choice. It is more realistic to perform measurements involving only final state particles, and objects constructed from final state particles, such as dressed leptons \cite{Kar:ab1be6}.

The final state is defined solely in terms of final state particles as opposed to targeting a specific process. Beyond the requirement of two same flavour opposite sign lepton pairs, the measurement is inclusive to additional particles such as additional leptons, jets, and invisible particles. Previous irreducible backgrounds (\VVV, \ttZ) are now considered as part of the signal since they produce four or more prompt leptons.

\section{Background estimation}
\label{sec:background}

\section{Systematic uncertainties}
\label{sec:sysuncert}

%% Unfolding and respective studies
\section{Unfolding}
\label{sec:unfolding}

To unfold a measurement means to correct it for detector-effects, since what the detector sees is not what truly occurs in nature. Rather, data at the detector-level include additional side effects that one must consider, such as resolution effects, detector acceptance, etc. On the simulation side, detector simulations are the most computationally expensive. For re-interpretation purposes, having to run through the entire detector simulation chain for every model of interest is highly inefficient. Furthermore as detector technology advances and changes, as does the simulation software. In order to re-interpret detector level data published in a certain year, one would have to use the simulation software corresponding to what was in use back then. 

It is desirable to unfold Standard Model measurements so that it can be used and compared to theoretical predictions in years to come. By presenting measurements at the particle level, no further simulation is require beyond the raw output of the event generator, which is a big advantage. 

\subsection{Unfolding methodology}
\label{subsec:unfmethod}

\subsection{Binning optimization}
\label{subsec:binningopt}

The binnings of the measured distributions were optimized based on two factors: the number of events and the purity of each bin. Here the purity refers to the diagonal of the migration matrix normalised along truth, thus representing the fraction of truth events that end up in the same reconstructed event bin. There were several iterations of the binning that were run with varying criteria, summarised in table \ref{tab:BinningVersions}.

The first iteration of the binnings were run with the Default criteria. Here, depending on the number of events in the bin, the purity requirement varies. Bins with lower statistics have a high purity requirement to reduce bin-to-bin migrations. The minimum number of events required for each bin is 10. Between 10 and 15 events, the purity was required to be at least 80\%. Between 15 and 20 events the purity must be 70\% or higher. Finally for bins with more than 20 events the purity cut was 60\%. 

The binning algorithm is as follows. For the full \mFourL differential mass distribution from \unit{20}{\Gev} - \unit{2000}{\GeV}, the distribution was first split into very fine steps of \unit{1}{\GeV} bins from \unit{20}{\Gev}-\unit{450}{\GeV}. From \unit{450}{\Gev}-\unit{2000}{\GeV} wider steps of \unit{5}{\GeV} bins were used. Due to the fine nature of the bin widths, this initial binning failed to meet any of the binning criteria. Next, the binning algorithm starts from the low mass end and starts to merge adjacent bins together if the criteria were not met. For example, if bin number 1 [20,21] has > 10 events, the algorithm merges bin number 1 with the next bin. The new bin number 1 is now [20,22]. Once again, if this bin has > 10 events, it will merge again and become [20,23], and so on and so forth until 10 events has been reached. Of course the purity must also pass the required percentage for the number of events in the bin, otherwise further bin merging occurs.  

Next we have the \mFourL distributions in double differential slices of \ptFourL, \yFourL, and flavour channel. For these distributions, the fine binning is defined as the the binning of the full \mFourL differential mass distribution, i.e. the output of the algorithm described in the previous paragraph. Bins were once again checked for number events and purity, and merged as needed. This was implemented so that all \mFourL in each of the  \ptFourL, \yFourL, and flavour slices would have bin edges that match with the inclusive distribution. 

For the distributions measured double differentially in the four \mFourL regions corresponding to \Z, \Higgs, On-shell \ZZ, and Off-shell \ZZ, the same procedure was followed for binning optimisation. Each distribution had a fine binning defined, and the bins were merged from left to right of the x-axis until the criteria were met. 

\begin{table}[bp]
  \begin{tabular}{lllll}
                & Default              & Stringent              & High statistics             \\
    \midrule
                                & 10 (purity > 0.8)   & 14 &   \\
     Minimum number of events & 15 (purity > 0.7) & 20 & 100    \\
                                &20 (purity > 0.6) & 25 &    \\
  \end{tabular}
  \caption{Three different versions of binning with varying criteria.}
  \label{tab:BinningVersions}
\end{table}

\subsection{Pre-unfolding weights}
\label{subsec:preuf}



\subsection{Injection tests}
\label{subsec:injection}


\section{Results}
\label{sec:results}

Results plots