\usepackage{xspace}
\usepackage{tikz}
\usepackage{morefloats,afterpage}
\usepackage{mathrsfs} % script font
\usepackage{verbatim}
\usepackage{caption}
\usepackage{subcaption}

\usepackage[acronym]{glossaries}
\makeglossaries

%% Feynman diagrams
\usepackage{tikz}
\usepackage[compat=1.1.0]{tikz-feynman}


%% To-do notes
\setlength{\marginparwidth}{2cm}
\usepackage[colorinlistoftodos,prependcaption,textsize=tiny]{todonotes}
\DeclareRobustCommand{\unsure}[2][1=]{\todo[linecolor=red,backgroundcolor=red!25,bordercolor=red,#1]{#2}}
\DeclareRobustCommand{\change}[2][1=]{\todo[linecolor=blue,backgroundcolor=blue!25,bordercolor=blue,#1]{#2}}
\DeclareRobustCommand{\info}[2][1=]{\todo[linecolor=green,backgroundcolor=green!25,bordercolor=green,#1]{#2}}
\DeclareRobustCommand{\improvement}[2][1=]{\todo[linecolor=purple,backgroundcolor=purple!25,bordercolor=purple,#1]{#2}}

%% Using Babel allows other languages to be used and mixed-in easily
%\usepackage[ngerman,english]{babel}
\usepackage[english]{babel}
\selectlanguage{english}

%% Citation system tweaks
\usepackage{cite}
% \let\@OldCite\cite
% \renewcommand{\cite}[1]{\mbox{\!\!\!\@OldCite{#1}}}

%% Maths
% TODO: rework or eliminate maybemath
\usepackage{abmath}
\DeclareRobustCommand{\mymath}[1]{\ensuremath{\maybebmsf{#1}}}
% \DeclareRobustCommand{\parenths}[1]{\mymath{\left({#1}\right)}\xspace}
% \DeclareRobustCommand{\braces}[1]{\mymath{\left\{{#1}\right\}}\xspace}
% \DeclareRobustCommand{\angles}[1]{\mymath{\left\langle{#1}\right\rangle}\xspace}
% \DeclareRobustCommand{\sqbracs}[1]{\mymath{\left[{#1}\right]}\xspace}
% \DeclareRobustCommand{\mods}[1]{\mymath{\left\lvert{#1}\right\rvert}\xspace}
% \DeclareRobustCommand{\modsq}[1]{\mymath{\mods{#1}^2}\xspace}
% \DeclareRobustCommand{\dblmods}[1]{\mymath{\left\lVert{#1}\right\rVert}\xspace}
% \DeclareRobustCommand{\expOf}[1]{\mymath{\exp{\!\parenths{#1}}}\xspace}
% \DeclareRobustCommand{\eexp}[1]{\mymath{e^{#1}}\xspace}
% \DeclareRobustCommand{\plusquad}{\mymath{\oplus}\xspace}
% \DeclareRobustCommand{\logOf}[1]{\mymath{\log\!\parenths{#1}}\xspace}
% \DeclareRobustCommand{\lnOf}[1]{\mymath{\ln\!\parenths{#1}}\xspace}
% \DeclareRobustCommand{\ofOrder}[1]{\mymath{\mathcal{O}\parenths{#1}}\xspace}
% \DeclareRobustCommand{\SOgroup}[1]{\mymath{\mathup{SO}\parenths{#1}}\xspace}
% \DeclareRobustCommand{\SUgroup}[1]{\mymath{\mathup{SU}\parenths{#1}}\xspace}
% \DeclareRobustCommand{\Ugroup}[1]{\mymath{\mathup{U}\parenths{#1}}\xspace}
% \DeclareRobustCommand{\I}[1]{\mymath{\mathrm{i}}\xspace}
% \DeclareRobustCommand{\colvector}[1]{\mymath{\begin{pmatrix}#1\end{pmatrix}}\xspace}
\DeclareRobustCommand{\Rate}{\mymath{\Gamma}\xspace}
\DeclareRobustCommand{\RateOf}[1]{\mymath{\Gamma}\parenths{#1}\xspace}

%% High-energy physics stuff
\usepackage{abhep}
\usepackage{hepnames}
\usepackage{hepunits}
\DeclareRobustCommand{\arXivCode}[1]{arXiv:#1}
\DeclareRobustCommand{\CP}{\ensuremath{\mathcal{CP}}\xspace}
\DeclareRobustCommand{\CPviolation}{\CP-violation\xspace}
\DeclareRobustCommand{\CPv}{\CPviolation}
\DeclareRobustCommand{\LHCb}{LHCb\xspace}
\DeclareRobustCommand{\LHC}{LHC\xspace}
\DeclareRobustCommand{\LEP}{LEP\xspace}
\DeclareRobustCommand{\CERN}{CERN\xspace}
\DeclareRobustCommand{\bphysics}{\Pbottom-physics\xspace}
\DeclareRobustCommand{\bhadron}{\Pbottom-hadron\xspace}
\DeclareRobustCommand{\Bmeson}{\PB-meson\xspace}
\DeclareRobustCommand{\bbaryon}{\Pbottom-baryon\xspace}
\DeclareRobustCommand{\Bdecay}{\PB-decay\xspace}
\DeclareRobustCommand{\bdecay}{\Pbottom-decay\xspace}
\DeclareRobustCommand{\BToKPi}{\HepProcess{ \PB \to \PK \Ppi }\xspace}
\DeclareRobustCommand{\BToPiPi}{\HepProcess{ \PB \to \Ppi \Ppi }\xspace}
\DeclareRobustCommand{\BToKK}{\HepProcess{ \PB \to \PK \PK }\xspace}
\DeclareRobustCommand{\BToRhoPi}{\HepProcess{ \PB \to \Prho \Ppi }\xspace}
\DeclareRobustCommand{\BToRhoRho}{\HepProcess{ \PB \to \Prho \Prho }\xspace}
\DeclareRobustCommand{\X}{\thesismath{X}\xspace}
\DeclareRobustCommand{\Xbar}{\thesismath{\overline{X}}\xspace}
\DeclareRobustCommand{\Xzero}{\HepGenParticle{X}{}{0}\xspace}
\DeclareRobustCommand{\Xzerobar}{\HepGenAntiParticle{X}{}{0}\xspace}
\DeclareRobustCommand{\epluseminus}{\Ppositron\!\Pelectron\xspace}
\DeclareRobustCommand{\protonproton}{\Pproton\Pantiproton\xspace}

\DeclareRobustCommand{\photon}{\gamma}


%% VLQ
%% Physics
\newcommand{\herwig}{H\protect\scalebox{0.8}{ERWIG}\xspace}
\newcommand{\rivet}{R\protect\scalebox{0.8}{IVET}\xspace}
\newcommand{\yoda}{Y\protect\scalebox{0.8}{ODA}\xspace}
\newcommand{\hepdata}{HEPData\xspace}
\newcommand{\contur}{\textsc{Contur}\xspace}


\newcommand{\pT}{\ensuremath{p_\mathrm{T}}\xspace}
\newcommand{\MET}{\ensuremath{E_\mathrm{T}^\mathrm{miss}}\xspace}
\newcommand{\CLs}{\ensuremath{\mathrm{CL}_\mathrm{s}}\xspace}

\newcommand{\B}{\ensuremath{B}\xspace}
\newcommand{\T}{\ensuremath{T}\xspace}
\newcommand{\X}{\ensuremath{X}\xspace}
\newcommand{\Y}{\ensuremath{Y}\xspace}
\newcommand{\BT}{\ensuremath{BT}\xspace}
\newcommand{\XT}{\ensuremath{X\mspace{-1mu}T}\xspace}
\newcommand{\BY}{\ensuremath{BY}\xspace}
\newcommand{\BTX}{\ensuremath{BT\mspace{-2mu}X}\xspace}
\newcommand{\BTY}{\ensuremath{BTY}\xspace}
\newcommand{\BTXY}{\ensuremath{BT\mspace{-2mu}XY}\xspace}

\newcommand{\triple}[3]{\text{#1:#2:#3}\xspace}
\newcommand{\WZH}{\triple{$W\!$}{$Z$}{$H$}}
\newcommand{\WZHozz}{\triple{1}{0}{0}}
\newcommand{\WZHzoz}{\triple{0}{1}{0}}
\newcommand{\WZHzzo}{\triple{0}{0}{1}}
\newcommand{\WZHooo}{\triple{1}{1}{1}}
\newcommand{\WZHttt}{\triple{$\tfrac{1}{3}$}{$\tfrac{1}{3}$}{$\tfrac{1}{3}$}}
\newcommand{\WZHtoo}{\triple{$\tfrac{1}{2}$}{$\tfrac{1}{4}$}{$\tfrac{1}{4}$}}
\newcommand{\WZHzoo}{\triple{0}{$\tfrac{1}{2}$}{$\tfrac{1}{2}$}}