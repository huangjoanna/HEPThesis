\chapter{Conclusion}
\label{chap:conclusion}

The Run-2 of the LHC at an unprecedented centre-of-mass energy of 13 TeV is now finished, and 139 fb-1 of data has been successfully collected. The ATLAS collaboration made great efforts in analyzing the data. In both making and refining cross-section measurements and conducting direct searches for new physics. As the latter continues to be unfruitful, the focus shifts to a complementary approach in the hunt for BSM physics - the reinterpretation of the precision measurements which are well compatible with the Standard Model. This approach is sensitive to potential new signatures that may be "hidden" in the data, appearing as subtle changes to a spectrum rather than a visible peak. 

Cross-sections measured in terms of final state particles have a high degree of model independence, facilitating its reuse in constraining BSM physics. One such precision measurement targets processes that have at least four leptons in the final state. Chapter~\ref{chap:m4l} presented the \ATLAS four-lepton cross-section measurements using \unit{139}{\invfb} of data. Four-lepton production has been measured inclusively, and differentially in the fiducial phase space, along with measurements of the dilepton mass, and kinematic and angular variables in four regions of \mFourL{} dominated by difference physics processes. All results are presented at the particle-level, meaning they are corrected for detector effects. The procedure used for the unfolding process was optimized and found to have minimal bias in the presence of physics beyond the SM. Overall, the measured distributions are found to be consistent with the leading SM predictions. A few interpretation examples are conducted by the analysis team. The most precise \ZFourL{} branching fraction to date was extracted and has a value of, consistent with previous measurements and predictions. Limits on coefficients in a SMEFT framework and parameters governing a model based on a spontaneously broken $B-L$ gauge symmetry were set using the data. To facilitate future re-interpretability, the data are made public and the analysis workflow documented. In doing so, the analysis team hopes that the measurement can be used to further probe BSM models in the years to come.

In Chapter~\ref{chap:reinterpretation}, two re-interpretation studies using re-interpretation software \contur are presented. The first focuses on a generic class of VLQ models, and the second is the $B-L$ model that was also interpreted within the \ATLAS \mFourL{} analysis. For these studies, a large bank of LHC differential cross-section measurements were used. Although these measurements were not specifically tailored to be sensitive to the BSM models, they nonetheless were able to exclude significant regions of the model parameter space. For the case of the VLQs, the fast-paced \contur machinery was able to survey previously unexplored model scenarios. The study of the $B-L$ model yielded competitive limits to that of the \ATLAS \mFourL{} analysis, and improves upon the limits of a previous paper with limits derived using the same toolkit~\cite{BLcontur}, thanks to the addition of new measurements. For both models, the role that the new \ATLAS \mFourL{} measurement plays was investigated. In certain scenarios, the addition of this new measurement resulted in a significantly larger excluded region. 

Overall, the work in this thesis demonstrated how a model-independent measurement is made, and how how powerful they can be, if well preserved, in constraining new physics. What takes a dedicated search team years accomplish, takes the \contur framework one day. Undeniably, this toolkit can be beneficial if used in complement by searches when designing analyses and deciding what parameter space to target. Furthermore, given how resource-intensive it is to publish an analysis, it is in the best interest of the particle physics community to maximize its usefulness. From an overarching collaborative perspective, ensuring an analysis' re-interpretability is highly desirable. It signifies a longer legacy for the published data well beyond the lifetime of the experiment. 


