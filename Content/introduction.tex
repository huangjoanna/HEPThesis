\chapter{Introduction}
\label{chap:intro}
What is the universe made of? This question has fascinated both philosophers and scientist alike for the past millenium. The best theory there is to date that answers this question is the Standard Model of Particle Physics, often abbreviated as the SM. All matter in the universe can be broken up into fundamental building blocks, the fundamental particles, which are govern by fundamental forces. The SM is the most accurate description that exists of how these particles and forces interact with one another. The construction of increasingly powerful particle colliders has led to the the observation of a plethora of new particles. The SM has been able to bring to order the observed particles, as well as accurately predict the existence of postulated particles. In 2012, two experiments (ATLAS and CMS) at the Large Hadron Collider (LHC) announced the discovery of a new particle exhibiting the predicted properties of the Higgs boson, thus completing the SM. Despite its immense success, however, the SM is not without imperfections. For example, the reasoning behind dark matter, dark energy, gravity, and the matter-antimatter asymmetry are not incorporated within the model. Rather, the explanations for these phenomena are postulated by theorists as Beyond the Standard Model (BSM) theories. 

Traditionally, the hunt for BSM physics at the LHC takes in the form of dedicated direct searches. A less labour-intensive, complementary probe for new physics involves exploiting the precision measurements made at the LHC of predicted SM processes. The work presented in this thesis focuses on the design and results of one such ATLAS precision measurement, and how it is used alongside other published LHC measurements to test new physics theories.

This thesis is structured as follows. Beginning with Chapter~\ref{chap:theory}, a brief review of the phenomenology of the Standard Model is given. Chapter~\ref{chap:ATLASdetector} explains the practical experimental aspects of the data used, including the LHC, the structure of the ATLAS detector, and how collision events are reconstructed. Chapter~\ref{chap:fourlepton} is the main focus of this thesis: the \ATLAS inclusive four-lepton analysis. The analysis design is described in detail, with particular attention paid to the unfolding procedure. The results using the complete \unit{139}{\invfb} Run II data-set are presented along with several interpretation examples. Finally, Chapter~\ref{chap:reinterpretation} introduces the concept of reinterpretation and presents the constraints that LHC measurements have on two BSM scenarios. 

\section{Author contributions}


