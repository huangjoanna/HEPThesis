\chapter{Introduction}
\label{chap:intro}
What is the universe made of? This question has fascinated both philosophers and scientist alike for the past millennium. The best theory there is to date that answers this question is the Standard Model of Particle Physics, often abbreviated as the SM. All matter in the universe can be broken up into fundamental building blocks, the fundamental particles, which are governed by fundamental forces. The SM is the most accurate description that exists of how these particles and forces interact with one another. The construction of increasingly powerful particle colliders has led to the observation of a plethora of new particles. The SM has been able to bring to order the observed particles, as well as accurately predict the existence of postulated particles. In 2012, two experiments (ATLAS and CMS) at the Large Hadron Collider (LHC) announced the discovery of a new particle exhibiting the predicted properties of the Higgs boson, thus completing the SM. Despite its immense success, however, the SM is not without imperfections. For example, the reasoning behind dark matter, dark energy, gravity, and the matter-antimatter asymmetry (i.e. the excess of baryons over anti-baryons) are not incorporated within the model. Rather, the explanations for these phenomena are postulated by theorists as Beyond the Standard Model (BSM) theories. 

Traditionally, the hunt for BSM physics at the LHC takes in the form of dedicated direct searches. A less labour-intensive, complementary probe for new physics involves exploiting the precision measurements made at the LHC of predicted SM processes. The work presented in this thesis focuses on the design and results of one such ATLAS precision measurement, and how it is used alongside other published LHC measurements to test new physics theories.

This thesis is structured as follows. Beginning with Chapter~\ref{chap:theory}, a brief review of the phenomenology of the Standard Model is given. Chapter~\ref{chap:ATLASdetector} explains the practical experimental aspects of the data used, including the LHC, the structure of the ATLAS detector, and how collision events are reconstructed. Chapter~\ref{chap:fourlepton} is the main focus of this thesis: the \ATLAS inclusive four-lepton analysis. The analysis design is described in detail, with particular attention paid to the unfolding procedure which corrects the data for detector effects. The results using the complete \unit{139}{\invfb} Run II data-set are presented along with several interpretation examples. Finally, Chapter~\ref{chap:reinterpretation} introduces the concept of reinterpretation and presents the constraints that LHC measurements have on two BSM scenarios. 

\subsubsection{Author contributions}
The analysis presented in of Chapter~\ref{chap:fourlepton} is performed as part of the \ATLAS Collaboration with contributions from multiple international institutes. The results presented relies on the work of many engineers, technicians, experimentalists, and theorists alike. The data analysis itself is a sub-group effort, and would not have been possible without the hard work from every team member. The specific contributions from the author to the analysis are as follows:
\begin{itemize}
    \setlength\itemsep{-0.6em}
    \item Validations studies of the Monte Carlo samples
    \item Binning definition of all observables
    \item Creation of unfolding inputs
    \item Monte Carlo closure tests
    \item Signal injection studies
    \item Unfolding the data and presentation of final results.
\end{itemize}
The author did not directly work on the evaluation of the background estimation, the evaluation of the uncertainties, or the in-house interpretations.

The VLQ re-interpretation study of Chapter~\ref{chap:reinterpretation} is a collaborative effort between UCL and the University of Glasgow with team members Andy Buckley, Jon Butterworth, Louie Corpe, and Puwen Sun. The results of the study is published in Reference~\cite{VLQ_contur}. Additionally presented are some updated \contur scans performed by the author, and an investigation the role of the new \ATLAS four-lepton analysis. The \contur study on the gauged $B-L$ model of Chapter~\ref{chap:reinterpretation} was conducted by the author, following guidance from Reference~\cite{BLcontur} regarding what regions of parameter space to sample. The results are compared to the \ATLAS limits, which the author did not work on.

The \ATLAS author qualification task conducted by the author is not included explicitly in this thesis, however it is documented in Reference~\cite{Huang:2676143}. 