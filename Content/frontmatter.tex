%% Title
\titlepage[]{%
  A dissertation submitted to University College London\\ for the degree of Doctor of Philosophy}

%% Abstract
\begin{abstract}%[\smaller \thetitle\\ \vspace*{1cm} \smaller {\theauthor}]
  %\thispagestyle{empty}
  This thesis describes the analysis design and results of the \ATLAS four-lepton measurement, using 139~fb$^{-1}$ of data collected in 13~TeV proton-proton collisions at the Large Hadron Collider. The measurement is designed for maximal model-independence and inclusivity. Defined solely in terms of the final state particles, any process leading to the creation of four or more leptons is considered to be a part of the signal. The results are presented in the form of inclusive and fiducial differential cross-sections, and are corrected for detector effects via an iterative Bayesian technique. The measurement is compared to state-of-the-art Standard Model predictions, and the two are found to be consistent. Secondly, two re-interpretation studies are presented where existing precision fiducial measurements, including the aforementioned four-lepton measurement, are used to set constraints on two beyond the Standard Model theories. The first is a generic model of vector-like quarks, and the second is a model with a gauged and spontaneously broken $B-L$ symmetry. These studies are conducted using the \contur re-interpretation toolkit. The derived limits are competitive with existing \ATLAS limits, and exclude previously unexplored regions of parameter space. 
\end{abstract}

\chapter*{\centering Impact Statement}
The research presented in this work plays a part in the continuous efforts of the particle physics community in refining the Standard Model, as well as searching for what lies beyond it. Using new LHC data collected by the ATLAS detector during Run 2, a measurement was made for events with a specific final state. Steps were taken to ensure the data are well preserved such that they may be re-used by the theoretical community. Two studies were performed to demonstrate the power of such re-use. The work carried out in this thesis is an advocate for re-interpretation. In maximizing the impact of the published data for years to come like so, researchers will be able to push the frontiers of particle physics even further. Outside of the academic world, the analysis techniques and experimental technologies developed have wide applications that improve and facilitate humans' quality of life. Particle accelerators are now used by most major medical centres for treatment and diagnosis, and the use of batch systems to handle unprecedented amounts of data is now adopted by many government and private industries. 


%% Declaration
\begin{declaration}
  I confirm that the work presented in this thesis is my own. Where information has been derived from other sources, I confirm that this has been indicated in the thesis.
  \vspace*{1cm}
  \begin{flushright}
    Dan Ping (Joanna) Huang
  \end{flushright}
\end{declaration}


%% Acknowledgements
\begin{acknowledgements}
    There are a large number of people who have provided me with tremendous support (either in my research or my personal life) during my time at UCL. First, I would like to thank my supervisor, Jon Butterworth, for offering me such a wonderful opportunity, and for guiding me along the research of this thesis. I am genuinely grateful for your constant encouragement, feedback, and support. None of this would have been possible without you. I extend my thanks to MCnet and the European Union’s Horizon 2020 research and innovation programme under the Marie Sklodowska-Curie grant for funding this work. 
    
    A big thank you goes to the talented members of the ATLAS four-lepton analysis team for your support and patience with my work. Special thanks goes to to Zara Grout, Max Goblirsch, and Emily Nurse, who were the best mentors I could ask while working on the analysis. I am very thankful to Andy Buckley, Louie Corpe, and Puwen Sun. It was an absolute privilege working along your sides for the vector-like quarks project. Another thank you goes to the brilliant minds of the many \contur developers. In particular, thank you to David Yallup for your invaluable mentor-ship and friendship. 
    
    To the many UCL HEP students who have become my dear friends, thank you. Life in the office was so much more lively (and possibly less productive) because of you lot. I feel so lucky and inspired to be surrounded by so many amazing people. Thank for you the Friday post-work pub sessions, for the ridiculous office banter, and for the non-stop flow of memes throughout the pandemic to liven the mood. I am very grateful to Seb for the countless coffees and hugs. In retrospect, I wished I used the “one favour” coupon to ask you to write this thesis. Thank you to Caishan, whose smile never fails to be a ray of sunshine in my day, and who I can count on to bring my passport to the airport.
    
    To my incredible group of friends that I met through the Imperial Outdoors club, thank you for the countless happy memories and becoming my second family. Thank you for roping me into climbing and taking me on my best, and scariest, adventures. Aina, your support and wisdom has seen me through hurdle after hurdle, and you've taught me the versatility of cooking with onions, thank you. Dimitris, thank you for cheering me on (for work, climbs, and life), and for showing me how stylish an oversized fanny pack can be. Micol, thank you for always being so incredibly kind and caring to me, and for teaching me what good olive oil should taste like. Tom, thank you for having my back no matter what. I look forward to when we visit Chick’n. 
    
    Adélie, my deepest thank you for being so endlessly kind, and for being my rock throughout the last few years. You’re worth more than diamonds more than gold. 
    
    Rasa, you’ve been with me through every obstacle. From the bottom of my heart, thank you for being more dedicated to our friendship than Drake is to his facial hair. 
    
    A profound thank you to Arthur for always giving me strength when I needed it, and for having a black hole as a stomach so I can cook endlessly to my heart’s content. 
    
    My two little nuggets of joy, Aaron and Aiden, thank you for existing. If you two read this in the future, go and give your big sister a hug.
    
    Finally, a most sincere thank you to my mom and dad, for all the hardship you have endured so that I can be where I am today. Thank you for always for supporting me through and through. 
\end{acknowledgements}

%% ToC
\tableofcontents

%% Strictly optional!
% \frontquote{%
%   Don't leave your thesis until the last minute Joanna!!}%
%   {Yourself}
% %% I don't want a page number on the following blank page either.
% \thispagestyle{empty}
