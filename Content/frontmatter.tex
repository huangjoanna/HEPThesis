%% Title
\titlepage[]{%
  A dissertation submitted to University College London\\ for the degree of Doctor of Philosophy}

%% Abstract
\begin{abstract}%[\smaller \thetitle\\ \vspace*{1cm} \smaller {\theauthor}]
  %\thispagestyle{empty}
  First off, this thesis describes the analysis design and results of the \ATLAS four-lepton measurement, using 139~fb$^{-1}$ of data collected in 13~TeV proton-proton collisions at the Large Hadron Collider. The measurement is designed for maximal model-independence and inclusivity. Defined solely in terms of the final state particles, any process leading to the creation of four or more leptons is considered to be a part of the signal. The results are presented in the form of inclusive and fiducial differential cross-sections, and are corrected for detector effects via an iterative Bayesian technique. The measurement is compared to state-of-the-art Standard Model predictions, and the two are found to be consistent. Secondly, two re-interpretation studies are presented where existing precision fiducial measurements, including the aforementioned four-lepton measurement, are used to set constraints on two beyond the Standard Model theories. The first is a generic model of vector-like quarks, and the second is a model with a gauged and spontaneously broken $B-L$ symmetry. These studies are conducted using the \contur re-interpretation toolkit. The derived limits are competitive with existing \ATLAS limits, and exclude previously unexplored regions of parameter space. 
\end{abstract}

\chapter*{\centering Impact Statement}
The research presented in this work plays a part in the continuous efforts of the particle physics community in refining the Standard Model, as well as searching for what lies beyond it. Using new LHC data collected by the ATLAS detector during Run 2, a measurement was made for events with a specific final state. Steps were taken to ensure the data are well preserved such that they may be re-used by the theoretical community. Two studies were performed to demonstrate the power of such re-use. The work carried out in this thesis is an advocate for re-interpretation. In maximizing the impact of the published data for years to come like so, researchers will be able to push the frontiers of particle physics even further. Outside of the academic world, the analysis techniques and experimental technologies developed have wide applications that improve and facilitate humans' quality of life. Particle accelerators are now used by most major medical centres for treatment and diagnosis, and the use of batch systems to handle unprecedented amounts of data is now adopted by many government and private industries. 


%% Declaration
\begin{declaration}
  I confirm that the work presented in this thesis is my own. Where information has been derived from other sources, I confirm that this has been indicated in the thesis.
  \vspace*{1cm}
  \begin{flushright}
    Dan Ping (Joanna) Huang
  \end{flushright}
\end{declaration}


%% Acknowledgements
\begin{acknowledgements}
  Lots of people to thank...
\end{acknowledgements}

%% ToC
\tableofcontents

%% Strictly optional!
% \frontquote{%
%   Don't leave your thesis until the last minute Joanna!!}%
%   {Yourself}
% %% I don't want a page number on the following blank page either.
% \thispagestyle{empty}
