%% Theoretical predictions 
\section{Monte Carlo predictions}
\label{sec:montecarlopred}
Monte Carlo simulations are used to model the signal processes at detector-level and particle-level for this analysis, and to construct the response matrices that correct the data for detector effects. This section provides a description of the event samples that are used for this analysis in the standard description of the ATLAS collaboration. \todo[inline]{Add MCEG section if time permits??} %The general principles of Monte Carlo event generators are discussed briefly in Appendix \ref{chap:mcevgen}.

\subsection{\qqFourL}
The dominant \qqFourL process is simulated using the \SHERPA {2.2.2} event generator~\cite{Bothmann:2019yzt} with the \nnpdfnnlo{} set of PDFs~\cite{Ball:2014uwa}. The matrix elements are calculated at next-to-leading order accuracy for final states with zero and one jet, and at leading order accuracy for two- and three-jet final states. The different parton multiplicities are merged together and matched to the \SHERPA parton shower model based on the Catani-Seymour dipole factorization~\cite{Gleisberg:2008fv,Schumann:2007mg} using the MEPS@NLO prescription~\cite{Hoeche:2011fd,Hoeche:2012yf,Catani:2001cc,Hoeche:2009r}. A dedicated set of tuned parton-shower parameters developed by the \SHERPA authors are used. 
An alternate sample of the \qqFourL process is generated using  \POWHEGBOX v2~\cite{Alioli:2010xd,Melia:2011tj,Nason:2013ydw}. The sample is generated at NLO accuracy and interfaced to \PYTHIA 8.186 for the simulation of the parton shower, hadronization, and underlying event. The tuning parameters are set according to the AZNLO tune~\cite{STDM-2012-23}. The sample is corrected to higher order effects using a k-factor obtained with \textsc{Matrix} NNLO QCD prediction~\cite{Cascioli:2014yka,Grazzini:2015hta,Grazzini:2017mhc,Kallweit:2018nyv}. The $K$-factor is defined as the ratio of the NNLO cross-section to the NLO cross-section and applied as a function of \mFourL{}. 
Virtual electroweak NLO effects are accounted for by reweighting both samples with a mass-dependent $K$-factor. The high-order real electroweak contribution of $ZZ$ plus two jets is modelled separately in a \SHERPA{} {2.2.2} sample. 

\subsection{\ggFourL{}}
The gluon-gluon initiated \ggFourL{} process is modelled by \SHERPA{} 2.2.2 at leading order QCD for up to one additional parton emission. The \SHERPA parton shower model based on the Catani-Seymour dipole factorisation is used. Also included in this sample is the s-channel Higgs signal \ggS{} and its interference with the SM box diagram, which has a sizeable contribution above \unit{130}{\GeV}. For particle-level masses below \unit{130}{\GeV} the sample includes on the \ggFourL box diagram because the role of interference is negligible. An NLO QCD $K$-factor is derived using the ratio of the \SHERPA{} sample to an MCFM NLO sample~\cite{Campbell:2011bn}. This is applied as a mass-dependent weight.
An additional constant $K$-factor of 1.2 is applied to account for NNLO effects on the off-shell Higgs production cross-section~\cite{Passarino:2013bha,deFlorian:2016spz}. The sample has a generator cut of $\mll > 10~\GeV$ for same-flavour, opposite-charge lepton pairs. The contribution is below this cut is accounted for through the reweighting to MCFM prediction. Scale and PDF uncertainties are derived in the same way as for the \SHERPA{} \qqFourL{} sample.

\subsection{On-shell Higgs}
The resonant Higgs-boson production is an important process and is generated independently using the most precise description available. The SM Higgs can be produced via gluon-gluon fusion, vector-boson fusion (VBF), Higgstrahlung ($VH$), and in association with a top quark pair ($t\overline{t}H$). The \texttt{PDF4LHC15nnlo} and \texttt{PDF4LHC15nlo} PDF set~\cite{Butterworth:2015oua} are used, alongside AZNLO tune for all on-shell Higgs samples. The dominant gluon–gluon fusion production channel is simulated using the \powheg{} NNLOPS program~\cite{Hamilton:2013fea,Hamilton:2015nsa,Alioli:2010xd,Nason:2004rx,Frixione:2007vw} at NNLO accuracy in QCD, and matched to \pythia~\cite{Sjostrand:2014zea} for the simulation of the parton shower and non-perturbative effects. The sample is normalized to N$^3$LO in QCD cross-sections, which has been calculated for the gluon-fusion process, and corrected for NLO electroweak effects~\cite{deFlorian:2016spz,Anastasiou:2016cez,Anastasiou:2015ema,Dulat:2018rbf,Harlander:2009mq,Harlander:2009bw,Harlander:2009my,Pak:2009dg,Actis:2008ug,Actis:2008ts,Bonetti:2018ukf}. 
\powheg~\cite{Nason:2009ai,Alioli:2010xd,Nason:2004rx,Frixione:2007vw} is interfaced to \pythia{} for the vector-boson fusion process, the $WH$ and $ZH$ production process, and the small contribution from associated productions with a $t\overline{t}$ pair. All are estimated with matrix elements up to NLO in QCD. For VBF, the prediction is reweighted to an approximate-NNLO QCD cross-section with NLO electroweak corrections~\cite{Ciccolini:2007jr,Ciccolini:2007ec,Bolzoni:2010xr}. For VH, the prediction is normalized to an NNLO QCD cross-section calculation with electroweak NLO corrections~\cite{Ciccolini:2003jy,Brein:2003wg,Brein:2011vx,Denner:2014cla,Brein:2012ne}. 
The uncertainties for the on-shell Higgs samples are identical of that of a previous Higgs analysis, the largest of which are from the QCD scale and PDF uncertainties. A detailed description can be found in Reference~\cite{HIGG-2016-22}. 
%  The simulation achieves NNLO accuracy for arbitrary inclusive $gg\to H$ observables by reweighting the Higgs boson rapidity spectrum in \texttt{Hj-MiNLO}~\cite{Hamilton:2012np,Campbell:2012am,Hamilton:2012rf} to that of HNNLO~\cite{Catani:2007vq}.

\subsection{$VVV$ and $t\overline{t}V(V)$}
A smaller contribution to the four-lepton final state originates from triboson processes, and vector-boson production in association of top-quark pairs. These are referred to as $VVV$ (for $WWZ, WZZ$ and $ZZZ$) and $t\overline{t}V(V)$ (for $t\overline{t}Z$ and $t\overline{t}WW$) respectively. The tribon processes are modelled with \SHERPA{} 2.2.2 at NLO accuracy in QCD, with a Catani–Seymour dipole factorization based parton shower provided by \SHERPA{}. Two samples are provided for the  $t\overline{t}V(V)$ contribution. The first is simulated with \SHERPA{} 2.2.0 at LO accuracy up to final states with one additional jet. This sample is used to construct the response matrix used to correct the data for detector effects. The second prediction is produced with the \madgraph~2.3.3~\cite{Alwall:2014hca} generator at NLO accuracy interfaced to \PYTHIAV{8.210}~\cite{Sjostrand:2014zea}. This particle-level predictions of this sample is used to compared against the data for the interpretations of Section \ref{sec:interpretations}. A flat uncertainty of $\pm15\%$ to account for the differences between the two samples is is assigned.

\subsection{Corrections}
All MC events are processed through GEANT4~\cite{Geant4} to simulate the expected response of the ATLAS detector. Next, the samples are passed through the same object reconstruction and identification algorithms as the data and the analysis selection is applied. An illustration of the interface between MC simulation and data analysis is shown in Figure \ref{fig:dataMCflow}. Pile-up is simulated with \PYTHIAV{8.186} as inclusive inelastic $pp$ collisions. The events are then reweighted to reproduce the distribution of the mean number of interactions per brunch crossing (33.7 on average for the whole dataset). Lastly, events are  reweighted to account for the differences of the lepton reconstruction, identification, isolation, and vertex-matching efficiencies between data and simulation.