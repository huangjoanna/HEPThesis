%% Signal definition and event selection
\section{Fiducial region}
\label{sec:signaldef}
The motivation behind this analysis is to make a measurement as inclusive and as model-independent as possible. Any process leading to a final state of four lepton - made up of two same flavour opposite sign electron or muon pairs - is considered to be part of the signal. Electrons or muons originating from fully leptonic decays of taus are counted towards the signal. This section describes the fiducial region definition of the analysis, summarized in Table~\ref{tab:fidregion}, which follows closely the acceptance of the detector. 

The signal is defined solely in terms of final state particles as opposed to targeting a specific process. Beyond the requirement of two same flavour opposite sign lepton pairs, the measurement is inclusive to additional particles such as additional leptons, jets, and invisible particles. Contributions from triboson production and vector boson production in association with top quarks are considered to be part of the signal since they produce four or more prompt leptons. In general, it is good practice to avoid the subtraction of so-called "irreducible backgrounds" (which invariably introduces a theory dependence) and instead make measurements defined solely in terms of the final state signature. 

For particle physicists, a prompt lepton simply means the lepton did not originate from a hadron. Prompt leptons are further classified into three categories depending on their association with emitted photons. These three categories are:
\begin{itemize}
    \item Born leptons: leptons prior to QED Final State Radiation (FSR);
    \item Bare leptons: leptons after QED FSR;
    \item Dressed leptons: leptons after QED FSR, that then have the four momenta of nearby radiated photons added to it. 
\end{itemize}
All experiments make lepton measurements after QED FSR has occurred. It is for this reason that born leptons are not the best choice. It is more realistic to perform measurements involving only final state particles, and objects constructed from final state particles, such as dressed leptons \cite{Kar:ab1be6}. In this analysis, a choice of dressing electrons but leaving muons bare was made to closer mimic what is seen by the detector. This choice is studied in detail in the internal note of the analysis~\cite{m4l_internalnote}. The leptons are required to satisfy $p_T$ and $|\eta|$ requirements motivated by detector acceptance. For electrons, the $p_T$ must be greater than \unit{7}{\GeV} and $|\eta|<2.47$. Muons must have $p_T > $\unit{5}{\GeV} and $|\eta|<2.7$.

Events must contain a minimum of four leptons formed of two same-flavour, opposite-sign electron or muon pairs. Additional requirements are set on the transverse momentum of the leading and sub-leading lepton to be higher than \unit{20}{\GeV} and \unit{10}{\GeV} respectively. A lepton angular separation requirement of $\Delta R > 0.05$ is applied to all leptons. A cut on the invariant mass of all SFOS lepton pairs is made at \unit{5}{\GeV}. The motivation behind these cuts and described in Section~\ref{sec:eventselection}.

%% Truth isolation
Finally, an emulation of the reconstruction-level isolation criteria \todo[]{What is this isolation criteria?} is included in the fiducial region definition. Although the particle-level application is a simplification, it nevertheless returns a result that is closer to what is actually measured. The particle-level truth isolation criteria requires the sum of the transverse momentum of all charged particles inside a $\Delta R  = 0.3$ cone of the lepton, divided by transverse momentum of the lepton itself, to be less than 0.16. If any other selected leptons are within the cone, their momenta is not included. 

\begin{table}[t]
  \begin{tabular}{lllll}
    \toprule
    Electrons               & Dressed, $p_T$ > \unit{7}{\GeV}, $|\eta| < 2.47$\\ 
    Muons                   & Bare, $p_T$ > \unit{5}{\GeV}, $|\eta| < 2.7$\\
    \bottomrule
    \toprule
    Four-lepton signature   & Minimum four leptons\\
                            & Two same-flavour, opposite-sign electron or muon pairs\\
    Lepton kinematics        &   Leading lepton $p_T > 20$~\GeV{}  \\
                             & Sub-leading lepton $p_T > 10$~\GeV{}  \\  [0.3cm]
    Lepton separation        &   $\Delta R > 0.05$ between all leptons \\
    $J/\psi$-Veto           &    $  m_{\ell\ell} > 5$~\GeV{} for all SFOS pairs \\
    Truth isolation         & $\dfrac{p_T(\Delta R  = 0.3)}{p_T(\text{lepton})}<0.16$ \\
    \bottomrule
    \toprule
    Quadruplet selection    & The two SFOS pairs closest to $m_Z$ are assigned as \\
                            & the primary and secondary pair \\
    \bottomrule
  \end{tabular}
  \caption{Fiducial region definition.}
  \label{tab:fidregion}
\end{table}

\section{Measured observables}

The star observable of the analysis is none other than the four lepton invariant mass, \mFourL. It has been measured previously by both the \ATLAS and the \CMS experiment~\cite{Aad:2014tca,Aad:2015rka,Aaboud:2019lxo}. As with the previous round of the analysis, the \mFourL{} distribution is also measured double-differentially, in slices of the transverse momentum of the four lepton system, the absolute rapidity of the four lepton system, and the flavour channel of the four lepton system. 

New to this round of the analysis is the division of the four lepton invariant mass spectrum into four separate regions, each dominated by a different process. From \unit{60}{\GeV}-\unit{100}{\GeV} resonant single \Z production reigns, similarly the \unit{120}{\GeV}-\unit{130}{\GeV} region is dominated by Higgs production, and the high mass region from \unit{180}{\GeV}-\unit{2000}{\GeV} by \onshellZZ{} production. Lastly to fill the gaps between  \unit{20}{\GeV}-\unit{60}{\GeV}, \unit{100}{\GeV}-\unit{120}{\GeV}, and \unit{130}{\GeV}-\unit{180}{\GeV} is the \offshellZZ{} region. This is summarized in Table \ref{tab:m4lregions}. 

The measured distributions as a function of \mFourL{} are:
\begin{itemize}
    \item Inclusive \mFourL{};
    \item \mFourL{} in slices of the four-lepton quadruplet transverse momentum, \ptFourL{};
    \item \mFourL{} in slices of the absolute rapidity of the quadruplet \yFourL{};
    \item \mFourL{} in the decay channels $4e, 4\mu, $ and $2e2\mu$.
\end{itemize}
The following variables are measured double differentially in the four \mFourL{} regions:
\begin{itemize}
    \item Cosine of angle $\theta^{*}$, where $\theta^{*}$ is the angle between the negative lepton in the di-lepton rest frame (leading pair) and the di-lepton pair in the lab frame (sub-leading pair). This angle is sensitive to the polarization of the decaying boson. This is measured for the primary and secondary lepton pair, \CTSOneTwo and \CTSThreeFour
    \item The difference in rapidity between the lepton pairs \dYPairs
    \item The difference in azimuthal angle between the lepton pairs, and between leading leptons, \dPhiPairs, \dPhill
    \item The invariant mass of the lepton pairs \mZOne and \mZTwo
    \item The transverse momentum of the lepton pairs \ptZOne and \ptZTwo
\end{itemize}

\begin{table}[t]
  \begin{tabular}{lllll}
        Region & \mFourL interval(s) \\
        \midrule
        \ZFourL & \unit{60}{\GeV} < \mFourL < \unit{100}{\GeV} \\
        \HFourL & \unit{120}{\GeV} < \mFourL < \unit{130}{\GeV} \\
        \onshellZZ{} & \unit{180}{\GeV} < \mFourL < \unit{2000}{\GeV} \\
        \offshellZZ{} & \unit{20}{\GeV} < \mFourL < \unit{60}{\GeV}, \unit{100}{\GeV} < \mFourL < \unit{120}{\GeV}, \\
          & and \unit{130}{\GeV} < \mFourL < \unit{180}{\GeV}\\
  \end{tabular}
  \caption{The four \mFourL regions dominated by the single \Z, Higgs, \onshellZZ{} and \offshellZZ{} processes.}
  \label{tab:m4lregions}
\end{table}
