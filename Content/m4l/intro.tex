%% m4l motivation

\section{Motivation for the \mFourL{} measurement}
\label{sec:fourlepmotivation}

The four lepton final state is a particularly interesting channel to study as it receives contributions from many physics processes. The predicted cross-sections for these processes are shown as a function of the invariant four-lepton mass in Figure~\ref{fig:m4lbreakdown}. First and foremost, there is the production of a pair of $Z$-bosons via quark-antiquark interactions in both the $t$- and $u$-channel, drawn in red in Figure~\ref{fig:m4lbreakdown}. The $t$-channel diagram is shown in Figure \ref{fig:m4lfeynman:qqZZ}, and represents, by far, the largest contribution to the $ZZ$ production and thus to the \mFourL{} distribution. At low masses where $\mFourL{}=m_{Z}$, single resonant \ZFourL{} production through QED radiative processes leads to a peak in the spectrum via the $s$-channel diagram in Figure \ref{fig:m4lfeynman:singleZ}. At $\mFourL{}=\unit{180}{\GeV}$ and beyond, the threshold for the on-shell production of two $Z$ bosons is reached and results in a peak in the four lepton invariant mass spectrum. 

Second in magnitude is the gluon-induced production of a $Z$ boson pair as shown in dark blue in Figure~\ref{fig:m4lbreakdown}, and the corresponding Feynman diagram is shown in Figure \ref{fig:m4lfeynman:ggZZ}. This occurs via a triangle or box quark loop, which results in a factor $\alpha_s^2$ suppression. It still plays a substantial role, however, because at small $x$\footnote{Here $x$ is the component of the proton's momentum carried by the struck quark. At the \LHC the protons have very high energies; therefore the \LHC can be described as a small $x$ collider \cite{zotov2012small}} gluon-gluon luminosity is higher than the quark-antiquark luminosity \cite{Glover:194539}. The contribution from this process in on the order of ten percent \cite{Becker:2230817}. 

Aside from $Z$ boson pair contributions, there is a small contribution from decaying Higgs bosons. The Higgs bosons are produced also via gluon fusion, as illustrated in Figure \ref{fig:m4lfeynman:ggHZZ}. There is resonant Higgs production at \mFourL=\unit{125}{\GeV}, and a non-resonant enhancement at $\mFourL{}=m_{t}=\unit{350}{\GeV}$ from the top quark loop. Beyond \unit{350}{\GeV}, the Higgs-mediated $Z$ boson pair production process destructively interferes with continuum production of on-shell $Z$ bosons \cite{Campbell_2016}. The Higgs contribution is illustrated by the cyan line in Figure~\ref{fig:m4lbreakdown}.

Finally there are small SM contributions from top-quark pair production in association with a dilepton pair (orange line in Figure~\ref{fig:m4lbreakdown}), from triboson processes where at least two bosons decay leptonically (purple line in Figure~\ref{fig:m4lbreakdown}), and from events where $\tau$-leptons decay to muons or electrons. 

The \mFourL{} distribution can be a useful probe for certain new physics scenarios. Take for example, the high mass tail of the invariant mass spectrum. This region is dependent on the couplings of the Higgs to incoming and outgoing particles while independent of the Higgs boson width \cite{Campbell_2016}, a unique property that can be exploited to derive model-independent limits on the Higgs couplings, and on the \todo{reword, this is copy pasted} contribution of new states in the Higgs to gluon coupling \cite{Cacciapaglia_2014}. It has also been previously exploited to derive model-independent constraints on the Higgs boson width \cite{Caola_2013}. 
%% Secondly, under specific assumptions a class of models exists for which the off-shell coupling measurement together with a measurement of the on-shell signal strength can be re-interpreted in terms of a bound on the total Higgs boson width. In this paper, we provide a first step towards a classification of the models for which a total width measurement is viable and we discuss examples of BSM models for which the off-shell coupling measurement can be important in either constraining or even discovering new physics in the upcoming LHC runs

A previous iteration of the \mFourL{} by the \ATLAS collaboration using \unit{36}{\invfb} of data can be found in Reference~\cite{Aaboud:2019lxo}. For the analysis presented in this chapter, the data used corresponds to \unit{139}{\invfb} at $\sqrt{s}=13$~TeV, including the \unit{36}{\invfb} of the previous iteration, collected by the ATLAS detector during Run 2 of the LHC between 2015 and 2018. Compared to the previous round, the new \mFourL{} measurement takes advantage of the increased data statistics and focuses on improving inclusivity and acceptance (particularly in the low mass region), and maximizing reinterpretability. Unlike the previous iteration of the four-lepton analyses and other dedicated $ZZ\rightarrow 4\ell$ analyses~\cite{Sirunyan:s10052-018-5567-9,Aaboud:97.032005}, the new inclusive \mFourL{} analysis does not have an upper mass cut on the lepton pairs. In addition, the lower $m_\ell\ell$ limit has been simplified to be 5 GeV for all lepton pairs. In the previous round of Reference~\cite{Aaboud:2019lxo}, the \mFourL{} mass lower limit was at 80 GeV. With improved statistics this is now at 20 GeV. The full set of changes in comparison to the previous analysis can be found in the internal note~\cite{m4l_internalnote}. 

This chapter presents the inclusive four-lepton measurement using \unit{139}{\invfb} of data in full, with a stronger focus on the unfolding studies on which the author contributed. The analysis is published in Reference~\cite{m4l2021_paper}, from which certain sections are adapted.

\begin{figure}
    \centering
    \includegraphics[width=0.7\textwidth]{Figures/m4l/processbreakdown.pdf}
    \caption{Breakdown of contributing processes in the \mFourL{} distribution as modelled by Monte Carlo simulation are shown in the solid coloured lines. The total of all processes is represented by the dotted black line.}
    \label{fig:m4lbreakdown}
\end{figure}

\begin{figure}
\centering
\begin{subfigure}{.24\textwidth}
  \centering
  \includegraphics[width=.99\textwidth]{Figures/FeynGraphs/qqZZ4l.pdf}
  \caption{\qqZZ}
  \label{fig:m4lfeynman:qqZZ}
\end{subfigure}%
\begin{subfigure}{.24\textwidth}
  \centering
  \includegraphics[width=.99\textwidth]{Figures/FeynGraphs/qqZZ4lrad.pdf}
  \caption{\ZFourL{}}
  \label{fig:m4lfeynman:singleZ}
\end{subfigure}
\begin{subfigure}{.24\textwidth}
  \centering
  \includegraphics[width=.99\textwidth]{Figures/FeynGraphs/ggZZ4lbox.pdf}
  \caption{\ggZZ}
  \label{fig:m4lfeynman:ggZZ}
\end{subfigure}
\begin{subfigure}{.24\textwidth}
  \centering
  \includegraphics[width=.99\textwidth]{Figures/FeynGraphs/ggZZ4lhiggs.pdf}
  \caption{\HZZFourL}
  \label{fig:m4lfeynman:ggHZZ}
\end{subfigure}
\caption{Feynman diagrams for quark- and gluon-induced $ZZ$ production. The processes shown are the main contributors. This figure is from Ref.~\cite{m4l2021_paper}.}
\label{fig:m4lfeynman}
\end{figure}

% this channel provides a clean leptonic final state resulting in a small instrumental background, where one or more of the reconstructed lepton candidates originate from the misidentification of jet fragments or from nonprompt leptons.
